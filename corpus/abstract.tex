%************************************************
% ABSTRACT
%************************************************

\begin{abstract}
    \paragraph{} Epilepsy is a neurological disorder characterized by epileptic seizures, which are episodes of vigorous shaking. People suffering from drug-resistant epilepsy cannot be treated and have to deal with epilepsy in everyday life. In recent years, the interest in the application of machine learning and deep learning models to the problem of seizure prediction has increased and several new approaches have been proposed. The anticipation of a seizure could give the time to implanted neurostimulators to intervene in advance and avoid the occurrence of the seizure; this is why machine learning application to this problem could be crucial. Previous approaches to this task achieved good results, but they often involve using ad-hoc models and relying on big electroencephalographic (EEG) datasets containing hundreds of seizures from several patients. Epileptic seizures data are significantly difficult to gather and the EEG monitoring causes discomfort to the patients, therefore the use of big datasets for the seizure prediction problem does not reflect a real-world scenario.

    This master thesis project presents a review of machine learning and deep learning methods and related performances for the epileptic seizure prediction task. The study is performed using intracranial electroencephalography data (IEEG) and it contains the implementation of a variety of machine learning and deep learning models, including the more recent and promising approaches for machine learning on graphs, in order to realize a comparison between all the approaches for the problem of seizure prediction. In this thesis we present our findings about the performance of the models we tested on the problem of seizure prediction and we propose some guidance on how to approach this problem in real-life situations, where few data are available. Unlike previous approaches, we intentionally used an extremely limited amount of data in order to test the models on a realistic scenario and to see if they could obtain good results even using few data to learn from.
\end{abstract}

%************************************************
% CONCLUSION
%************************************************

\chapter{Conclusion} \label{chap: conclusion}

\paragraph{} Through this project, we were able to provide to the reader a quite complete overview of machine learning and deep learning methods performances on the tasks of prediction of epileptic seizures.

We started by acquiring knowledge about the seizure prediction task and defining the three problem cases we chose to focus on. Consequently, we were provided with the data and we analyzed the dataset in order to familiarize with it; then we chose popular and commonly-used machine learning and deep learning models to test on this type of task. We decided to work with a severely limited amount of data in order to replicate a real-world scenario, where seizures data are highly difficult to gather. We preprocessed the data by shaping it to the necessary input format depending on the problem case and on the model. In order to test the models on the seizure prediction task, we ran several experiments for the various combinations of models and problem cases, looking for the models configurations that achieved the best performances. We made a comparison between the results of different models across the three problem cases and to identify the methods that worked better on each task. Thanks to the experiments, we were able to confirm of reject some assumptions we made at the beginning and to create new hypotheses based on the results.

By running several experiments and fine-tuning the models hyperparameters, we were able to predict a seizure up to 4 seconds in the future, achieving a recall of 0.767 on the test set, matched by high values for accuracy and ROC-AUC and a low loss. The model that was able to obtain these excellent performances is an \acs{lstm} neural network, which was also the best performer overall. The obtained results are noteworthy considering that the models has been trained and tested on a dataset containing only three seizures, representing an extremely challenging set-up for a such complex problem.

Our project demonstrated the ability of deep learning models to achieve good results on the seizure prediction task even in a situation characterized by a significant lack of data. It also showed that graph deep learning models, on the other hand, are not able to handle this extreme case scenario with the configuration proposed.

Future works on this project could involve a research on a bigger scale, testing a higher amount of machine learning and deep learning models and using some additional data preprocessing technique to improve performances, while still keeping the dataset fairly small to be more realistic. Another interesting deepening could be a study about how to create functional connectivity networks for this type of task, testing different connectivity metrics in order to identify the one that improves performances. 
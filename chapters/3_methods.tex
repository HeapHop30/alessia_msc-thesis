%************************************************
% METHODS
%************************************************

\chapter{Methods} \label{chap: methods}

\paragraph{} \textit{This chapter provides a high-level overview of the work that has been done for this project and the process that has been used.}

%------------------------------------------------
% Overview of the project steps
%------------------------------------------------

\section{Overview of the project steps} \label{sec: overview_project_steps}
\paragraph{} The work done for this Master Thesis project can be summarized via a list of steps we've gone through:
\begin{enumerate}
    \item Definition of the problem to solve;
    \item Research about previous approaches to the problem;
    \item Objective of the project;
    \item Data collection (out of the scope of this project);
    \item Data analysis and familiarization;
    \item Technical definition of the problem;
    \item Choice of the machine learning and deep learning models to use;
    \item Application of the models for the detection task;
    \item Application of the models for the prediction task;
    \item Results evaluation;
    \item Conclusions.
\end{enumerate}
The following sections will give a short description of what has been done for each project step.

%------------------------------------------------
% Definition of the problem to solve
%------------------------------------------------

\section{Definition of the problem to solve} \label{sec: step_definition_problem_to_solve}
\paragraph{} The problem we decided to focus on for this project is the seizure prediction problem, which consists of predicting in advance the happening of an epileptic seizure in a person. The study has been conducted using intracranial electroencephalography data (iEEG) and it is devoted to all the people suffering from drug-resistant epilepsy, which are forced to deal with epileptic seizures in everyday life. With the goal of being able to predict epileptic seizures few seconds in advance, the prediction would allow the neurostimulator to intervene in time on the patient brain in order to block the imminent occurrence of the seizure. The possibility to avoid even only a few of the epileptic seizures of a person would already improve by a lot her living standards.

%------------------------------------------------
% Research about previous approaches to the problem
%------------------------------------------------

\section{Research about previous approaches to the problem} \label{sec: step_research_about_previous_approaches}
\paragraph{} Once we identified the objective problem, we spent some time inquiring about previous approaches to the problem. With this aim, we read several scientific papers about seizure prediction in order to find out which methods have been already used and the level of performance they obtained. Through this research, we realized how complex was the problem and how difficult it was to obtain good results, even using powerful deep learning models. We also found out that, in order to have acceptable results, the majority of these models needed to be trained on a large amount of data, which in the case of epileptic seizures is difficult to gather.

%------------------------------------------------
% Objective of the project
%------------------------------------------------

\section{Objective of the project} \label{sec: step_objective_project}
\paragraph{} Having a clearer idea of the problem we were going to face and of its level of complexity, we defined the objective of the project and its boundaries. We decided to conduct a study on different machine learning and deep learning models applied to the problem of seizure prediction, generating a review of models to solve this problem and looking for the best configuration for each of them. The common thread we set throughout all the project was the lack of data to train the models. Epileptic seizure data, indeed, are very difficult to gather, so we wanted to find out how well the models would have performed having a very limited amount of data available to learn from.

%------------------------------------------------
% Data collection
%------------------------------------------------

\section{Data collection} \label{sec: step_data_collection}
\paragraph{} The data collection was out of our scope, since we were provided directly with ready-to-use data. The data collected for this project comes from a Toronto hospital and it consist of iEEGs generated from real measurements on a patient suffering from epilepsy. The data, corresponding to 24 hours of monitoring and containing three seizures, have been cleaned and provided to us in order to be used for this project.

%------------------------------------------------
% Data analysis and familiarization
%------------------------------------------------

\section{Data analysis and familiarization} \label{sec: step_data_analysis_familiarization}
\paragraph{} After having identified the problem and its difficulties, having defined the boundaries of the project and having retrieved the necessary data, we finally started with the actual implementation of the project. The first phase consisted only in a simple data analysis in order to familiarize with the data. We computed some basic statistical metrics on the iEEGs and we generated some plots of the data to understand it better. In this way, we identified the number of electrodes used for the measurements, the number of epileptic seizures available in the dataset and their duration and, through the plots, we could also visualize the behaviour of brain's electrical signals during the seizures.

%------------------------------------------------
% Technical definition of the problem
%------------------------------------------------

\section{Technical definition of the problem} \label{sec: step_technical_definition_problem}
\paragraph{} With some knowledge about the data, we could define the problem in a more technical and precise way. The technical definition of the problem has been already provided in Section \ref{sec: problem_definition}, where three cases have been presented: detection on a time step, detection on a sequence and prediction on a sequence respectively. During this project, all the three different cases has been addressed using suitable models depending on the task.

%----------------------------------------------------------------
% Choice of the machine learning and deep learning models to use
%----------------------------------------------------------------

\section{Choice of the machine learning and deep learning models to use} \label{sec: step_choice_models_to_use}
\paragraph{} For the choice of the machine learning and deep learning models to include in the review of methods, we wanted both to work with commonly used models and to apply suitable models for the seizure detection and prediction tasks. For these reasons, the choice fell on the following methods: Random Forest, Gradient Boosting and Support Vector Machine models representing the group of classic machine learning methods; Dense, Convolutional and LSTM neural networks representing the group of classic deep learning models; graph-based Convolutional and LSTM neural networks representing the group of graph-based deep learning models. Each model has been tested on the type of tasks that we though it fitted the most, based on the power and the complexity of the model.

%----------------------------------------------------------------
% Data preprocessing
%----------------------------------------------------------------

\section{Data preprocessing} \label{sec: step_data_preprocessing}
\paragraph{} The data has been preprocessed in order to be prepared to be used depending on the model and the task concerned. First thing, the dataset has been divided in a training set and a test set: the training set is represented by the data that the model analyzes in order to learn from it and to identify patterns, while the test data is useful to verify how much the trained model can generalize its knowledge to data that it never saw before. Since our dataset contained only three seizures, we used two seizures for the training set and the remaining one for the test set. Because of the lack of data, we couldn't create a validation set, so we mainly used the test set in order to evaluate the models. In the case of deep learning models, the training set has been standardized, since neural networks are very sensible to the numerical range of data features.

For the detection task on a single time step, no additional operations on data where required, since the dataset was already provided as a big sequence of time step - target couples. For this type of task, indeed, each sample consists of a single time step and the related target needs to be predicted.

For both the detection and prediction tasks on a sequence, on the other hands, sequences needed to be generated. The dataset has been converted in a set of sequences with a single target associated, with temporal relation between time steps inside the same sequence. Depending on whether we were working on detection or on prediction task, the target of each sequence corresponded to the target of the last time step in the sequence, or to the target of a future time step outside the sequence.

For the graph-based models, we transformed the samples from sequences of time steps to sequences of graphs.
\todo[inline]{Briefly describe how}

In order to have representative results, we performed k-folds cross validation for all the models and the tasks. Cross validation is a useful technique to evaluate machine learning models, especially when there is a limited amount of data at disposal. In order to perform k-folds cross validation, the dataset is divided into $k$ subsets (folds), of which $k-1$ folds are used as training set and the remaining fold as test set. The training-testing process is performed $k$ times, so that each fold is used as test set one time and it is included in the training set all the other times. To perform cross validation with our data, we divided the dataset into three subsets (3-folds cross validation), each one containing one of the three seizures, and we prepared the three corresponding training and test sets, each time using two folds as training set and the remaining fold as test set.

%----------------------------------------------------------------
% Application of the models for the detection task
%----------------------------------------------------------------

\section{Application of the models for the detection task} \label{sec: step_models_detection_task}
\paragraph{} gne

%----------------------------------------------------------------
% Application of the models for the prediction task
%----------------------------------------------------------------

\section{Application of the models for the prediction task} \label{sec: step_models_prediction_task}
\paragraph{} gne

%----------------------------------------------------------------
% Results evaluation
%----------------------------------------------------------------

\section{Results evaluation} \label{sec: step_restults_evaluation}
\paragraph{} gne

%----------------------------------------------------------------
% Conclusions
%----------------------------------------------------------------

\section{Conclusions} \label{sec: step_conclusions}
\paragraph{} gne
